\documentclass[12pt]{article}
\usepackage[english]{babel}
\usepackage{natbib}
\usepackage{url}
\usepackage[utf8x]{inputenc}
\usepackage{amsmath}
\usepackage{graphicx}
\graphicspath{{NITLOGO/}}
\usepackage{parskip}
\usepackage{fancyhdr}
\usepackage{vmargin}
\setmarginsrb{3 cm}{2.5 cm}{3 cm}{2.5 cm}{1 cm}{1.5 cm}{1 cm}{1.5 cm}

\title{DISRUPTIVE INNOVATIONS IN HEALTHCARE}\newline \\\\\\				
\author{21111006}								
\date{28 JAN 2022}						
\makeatletter
\let\thetitle\@title
\let\theauthor\@author
\let\thedate\@date
\makeatother

\pagestyle{fancy}
\fancyhf{}
\rhead{\theauthor}
\lhead{\thetitle}
\cfoot{\thepage}

\begin{document}
\begin{titlepage}
	\centering
    \includegraphics[scale = 0.22]{NITLOGO}\\[1.0 cm]	
    \textsc{\LARGE National Institute Of Technology \newline\\\\ RAIPUR}\\[2.0 CM]
    
	\textsc{\Large ASSIGNMENT 04}\\[0.5 cm]				% Course Code
	\rule{\linewidth}{0.4 mm} \\[0.4 cm]
	{ \huge \bfseries \thetitle}\\
	\rule{\linewidth}{0.4 mm} \\[1.5 cm]
	
	\begin{minipage}{0.6\textwidth}
		\begin{flushleft} \large
			\emph{Submitted To:}\\
			Mr. Saurabh Gupta\\
            Department Of Basic Biomedical Engineering\\
			\end{flushleft}
			\end{minipage}~
			\begin{minipage}{0.4\textwidth}
            
			\begin{flushright} \large
			\emph{Submitted By :}\\
			Akash Paikra\\
            21111006\\
		\end{flushright}
        
	\end{minipage}\\[2 cm]
\end{titlepage}

\tableofcontents
\pagebreak







\section{INTRODUCTION:}
Because every part of healthcare is dependent on some type of technology, technology is the most powerful driver of many disruptive breakthroughs in healthcare. Any new technology, from wearables and mobile phone apps to big data and artificial intelligence (AI) in diagnostics, has the potential to disrupt healthcare.

\subsection{ELECTRONICS HEALTH RECORDS (EHRs):}

Electronic health records have gotten a facelift over the years. With the IOT, big data and devices’ connectivity provide up-to-date information about a patient at their point of care. In our age of technology, the way we handle patient data is changing on all fronts. A healthcare provider can instantly pull up an EHRs and know a patient’s entire medical history. Imagine making an accurate diagnosis based on a patient’s past visits and suggesting the correct medication or procedure instantly.

EHRs make it easier for providers to communicate because they can be shared across devices and secure networks. Healthcare providers enjoy the ease with which EHRs allow them to update and expand on a patient's medical history. By giving an easily legible, comprehensive view into a patient's history, EHRs can also aid preventative care initiatives and avoid medical errors. A practitioner can examine a patient's medical history for trends that could indicate a medical issue that has gone unnoticed, and take early preventative action. The advancement of EHRs will only increase their value as a replacement for the obsolete and time-consuming physical recording method.



\subsection{REMOTE CARE:}
Many challenges in healthcare are being solved by remote patient care. Remote care relies on the IoT's convenience in moving data across devices to provide convenience while preserving high-quality treatment for patients. Video conferencing technology, big data, and wearable technology enable remote patient monitoring and telehealth. Physicians may watch and diagnose patients from afar, speaking with them about symptoms and even seeing medical issues in order to make an informed decision on medication or surgery without even being there in the room or state.

Wearable technologies and the Internet of Things enable remote patient monitoring, which means a physician may use data from a patient's watch to calculate their heart rate, caloric intake, and other factors that can help them avoid significant health problems. Additionally, Those with chronic illnesses can be monitored at home to maintain their independence and spend more time with their loved ones, only requiring medical attention when absolutely essential. Physicians will gain as well, as they will be able to counsel patients without having to travel.


Physicians can reach out to more people who don't like going to the doctor or can't afford it. Finally, by reducing hospital visits and freeing up hospital rooms for people who need them, as well as healthcare personnel' schedules, remote care can save money for everyone. Remote care will only improve in the future, disrupting healthcare with its promise of real-time communication, improved quality of life, more accessible health care, and cost savings.



\subsection{RETAIL CLINICS:}

Retail health clinics, often known as nurse-in-a-boxes, are walk-in clinics that can be found in retail stores, supermarkets, and pharmacies, such as CVS. Retail clinics are upending healthcare by giving people with convenient, high-quality care for minor ailments like allergies, colds and flu, as well as minor burns and sprains. EHRs are a critical component that allows retail clinics to operate.


Retail clinics, like remote care, are reducing travels to urgent care centres and other urgent hospital visits, freeing up healthcare professionals' time and allowing them to focus on providing quality care to people in need. Furthermore, retail clinics can reach a wider audience, and this increased accessibility, convenience, and cost reduction aims to address three significant issues in the healthcare industry today.



\subsection{AUGMENTED REALITY:}

Another developing technology impacting the healthcare business is augmented reality, which adds sights and sounds to reality to create its own sort of extended reality. Gaming, retail, and education are all utilising this new technology, with hospitals utilising augmented reality for educational purposes.


Consider how relaxed you'll feel going into an operation knowing your surgeon has done it many times before. Alternatively, imagine being a surgeon and going into an operation knowing exactly what to do. Both current physicians and trainees can learn how to execute treatments on a 3D representation of the human body using augmented reality. This enhanced circumstance places physicians in a low-risk training environment, allowing them to operate on virtual patients without the chance of failure.


This allows the physician to concentrate on the work at hand, allowing them to gain a deeper understanding of the situation and become better prepared to manage the real thing. Furthermore, augmented reality can be used to instruct a patient how to apply medication, bathe and dress a wound, and perform other tasks that can be performed by a patient rather than a doctor to prevent additional damage aggravation. Better education will lead to preventative health and a reduction in medical errors thanks to augmented reality.

\subsection{LASIK LASERS:}

Advances in laser technology have made it simple for doctors and more economical for individuals to forego eyeglasses and contacts in favour of a more permanent vision correction procedure. Another treatment where technology is assisting in cost reduction is vision correction, which is a key pain area in the healthcare business.


"Surgery is performed with specialised lasers that change the eye," according to today's most advanced LASIK lasers. A surgeon utilises these lasers to gently reshape the cornea during the surgery, correcting common vision abnormalities such as nearsightedness (myopia), farsightedness (hyperopia), and astigmatism." Because of the efficiency and safety of this elective treatment, surgeons may now provide services to a wider range of patients, making vision without corrective lenses more accessible and cheap.


















































\end{document}